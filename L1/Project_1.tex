\chapter{Lab 1: Basics} \label{day1}

The main goal of the first project in this course is to get in touch with the main concepts of the programming language ''VHDL`` and field-programmable gate arrays (FPGAs). 

First a simple program that connects the 16 switches of the ''Artix-7`` board to the corresponding LEDs was realised. Therefore ports for the switches and LEDs were added and connected via the constraints ''XDC`` file.

Then we created a simple logic function that compares 6 switches and outputs a signal to an LED. Here, we inspected the implementation of our program in the netlist and device view provided by the ''VIVADO`` editor.

Afterwards a parity generator was coded that outputs an 0 for an even number of bits and 1 for an odd one.


\section{Basic workflow}

\lstinputlisting[language=VHDL]{./L1/E1/src/project_1_1.vhd}

\section{Simple logic function}

\lstinputlisting[language=VHDL]{./L1/E2/src/project_1_2.vhd}

\section{Parity Generator}

\lstinputlisting[language=VHDL]{./L1/E3/src/project_1_3.vhd}

\section{Full adder}

\lstinputlisting[language=VHDL]{./L1/E4/src/project_1_4.vhd}

\section{Ripple-carry adder}

\lstinputlisting[language=VHDL]{./L1/E5/src/project_1_5.vhd}

\section{Inferring an adder from VHDL}

\lstinputlisting[language=VHDL]{./L1/E6/src/project_1_6.vhd}