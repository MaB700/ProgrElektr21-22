\chapter{Lab 2: Conditional Assignments} \label{day2}

In the second Lab we learn how to control the display of the \gls{fpga}, use entities and program a priority selection.

\section{Seven-segment digit: individual segments}

The seven-segment digit display can display four digits. Here, we only enabled the first digit by assigning "1110" to the anode value. The switches of the \gls{fpga} are connected to the cathode output and will control each of the seven-segments and the dot with one switch. 

\lstinputlisting[language=VHDL]{./L2/E1/src/project_2_1.vhd}

\section{Seven-segment digit: hexadecimal number}

The goal of this exercise is to display a single hexadecimal number on the seven-digit display. This number can be input with the switches on the \gls{fpga}. The entity ``project\_2\_2'' then selects the number for the input given by the switches accordingly. 

\lstinputlisting[language=VHDL]{./L2/E2/src/project_2_2_1.vhd}

\lstinputlisting[language=VHDL]{./L2/E2/src/project_2_2.vhd}

\section{Seven-segment digit: ASCII character}

The next challenge was to display an ASCII character. This could be done analogous to the method used in the previous exercise.

\lstinputlisting[language=VHDL]{./L2/E3/src/project_2_3_1.vhd}

\lstinputlisting[language=VHDL]{./L2/E3/src/project_2_3_2.vhd}

\section{Priority Encoder}

The priority encoder displays the number of the highest active input bit. All of the lower active input bits are ignored. The input is invalid if all of the input bits are 0, this is the only invalid input. If the output is invalid the VALID output goes to 1, otherwise to 0.

\lstinputlisting[language=VHDL]{./L2/E4/src/project_2_4.vhd}

\section{Seven-segment digit: mode selection}

This last exercise combines all of the above into a single program with a mode selection. The mode selection just selects the output on the display. Every single mode that can be selected is constantly evaluated by the \gls{fpga}. 

\lstinputlisting[language=VHDL]{./L2/E5/src/project_2_5x.vhd}
