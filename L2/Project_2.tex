\chapter{Lab 2: Conditional Assignments} \label{day2}

In the second Lab we learn how to control the display of the \gls{fpga}, use entities and programm a priority selection.

\section{Seven-segment digit: individual segments}

The seven-segment digit display can display four digits. Here, we only enabled the first digit by assigning "1110" to the anode value. The switches of the \gls{fpga} are connected to the cathode output and will control each of the seven-segments and the dot with one switch. 

\lstinputlisting[language=VHDL]{./L2/E1/src/project_2_1.vhd}

\section{Seven-segment digit: hexadecimal number}

The goal of this excercise is to display a single hexadecimal number on the seven-digit display. This number can be input with the switches on the \gls{fpga}. The entity ``project\_2\_2'' then selects the according number for the input given by the switches. 

\lstinputlisting[language=VHDL]{./L2/E2/src/project_2_2_1.vhd}

\lstinputlisting[language=VHDL]{./L2/E2/src/project_2_2.vhd}

\section{Seven-segment digit: ASCII character}

The next challenge was to display an ASCII character. This could be done anologous to the methode used in the previous excercise.

\lstinputlisting[language=VHDL]{./L2/E3/src/project_2_3_1.vhd}

\lstinputlisting[language=VHDL]{./L2/E3/src/project_2_3_2.vhd}

\section{Priority Encoder}

This priority encoder displays only the number of the highest input bit. If the first case is true the other cases will be ignored. The same is true for the ``VALID''-check. 

\lstinputlisting[language=VHDL]{./L2/E4/src/project_2_4.vhd}

\section{Seven-segment digit: mode selection}

This last excercise combines all of the above into a single programm with a mode selection. The mode selection just selects the output on the display. Every single mode that can be selected is constantly evaluated by the \gls{fpga}. 

\lstinputlisting[language=VHDL]{./L2/E5/src/project_2_5x.vhd}
