\chapter{Lab 9: Driving a VGA Monitor}

\section{Solid color}

This module is our main setup for the display. It includes all timings necessary to display a solid color on our display with HD+ resolution.

\lstinputlisting[language=VHDL]{./L9/E1/src/project_9_1.vhd}

\section{Patterns}

Here, we created one animated and three static patterns: an animated circle as well as a checkerboard, circles and stripes.

\lstinputlisting[language=VHDL]{./L9/E2/src/animated_circle.vhd}

\lstinputlisting[language=VHDL]{L9/E2/src/checkboard.vhd}

\lstinputlisting[language=VHDL]{L9/E2/src/circles.vhd}

\lstinputlisting[language=VHDL]{L9/E2/src/stripes.vhd}

\section{Video memory}

This module comes with a preloaded image that is saved to the ram of the \gls{fpga} and then displayed on a screen. Therefore we used the init$\_$ram function of VHDL from the instructions. Before that, the image has been converted to a black and white image to meet the RAM limitations (max. 1.800\,Kb) of the \gls{fpga}.

\lstinputlisting[language=VHDL]{L9/E3/src/project_9_3.vhd}

\lstinputlisting[language=VHDL]{L9/E3/src/video_memory.vhd}

\section{Video data from UART}

Here, we used UART to transmit data to the video memory of the \gls{fpga}. After converting a picture to a black and white image it can be transmitted with the terminal tool.

\lstinputlisting[language=VHDL]{L9/E4/src/project_9_1.vhd}